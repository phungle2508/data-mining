% !TEX program = pdflatex
% PowerPoint Slides - Cải Thiện Apriori
% Data Mining Project 2024-2025

\documentclass[aspectratio=169, 12pt]{beamer}

% Packages for Vietnamese language support
\usepackage[utf8]{inputenc}
\usepackage[T5]{fontenc}
\usepackage[vietnamese]{babel}

% Theme
\usetheme{Madrid}
\usecolortheme{whale}

% Fonts
\usepackage{mathptmx}
\usepackage{helvet}
\usepackage{courier}

% Other packages
\usepackage{graphicx}
\usepackage{booktabs}
\usepackage{array}
\usepackage{enumitem}
\usepackage{hyperref}
\usepackage{xcolor}
\usepackage{tikz}
\usetikzlibrary{shapes,arrows,positioning}

% Hyperlink setup
\hypersetup{
    colorlinks=true,
    linkcolor=cyan,
    urlcolor=cyan,
}
\setbeamertemplate{blocks}[default]
\setbeamercolor{block title}{bg=white, fg=black}
\setbeamercolor{block body}{bg=white}
\setbeamerfont{section in toc}{series=\bfseries}
\setbeamercolor{section in toc}{fg=black}% Metadata
\title[Cải Thiện Apriori]{CẢI THIỆN APRIORI}
\subtitle{Frequent Itemset Mining và Association Rules}
\author{Lê Đình Phùng -- 18130181 }
\date{Năm học 2025--2026}
\begin{document}

% =============================================================================
% SLIDE 1: TITLE SLIDE
% =============================================================================
\begin{frame}
\centering


% ===== Title =====
{\LARGE \textbf{BÁO CÁO}}\par
\vspace{1cm}

{\Large \textbf{CHỦ ĐỀ:} \textit{CẢI THIỆN THUẬT TOÁN APRIORI}}\par
\vspace{0.4cm}
{\Large \textbf{Môn học: Data Mining}}\par

\vspace{1.4cm}

% ===== Instructor =====
\begin{flushleft}
\textbf{Giảng viên phụ trách:} ThS. Trần Quốc Việt
\end{flushleft}



\vspace{1.2cm}

% ===== Student =====
\begin{flushleft}
\textbf{Sinh viên thực hiện:}\\
Lê Đình Phùng -- MSSV: 18130181
\end{flushleft}

\end{frame}


% =============================================================================
% SLIDE 2: TABLE OF CONTENTS
% =============================================================================
\begin{frame}{NỘI DUNG TRÌNH BÀY}
  \tableofcontents
\end{frame}

% =============================================================================
% SECTION 1: GIỚI THIỆU
% =============================================================================
\section{GIỚI THIỆU}

\begin{frame}{GIỚI THIỆU}
  \begin{block}{Market Basket Analysis là gì?}
    \begin{itemize}
      \item Kỹ thuật data mining trong lĩnh vực bán lẻ (retail)
      \item Phân tích hành vi mua sắm của khách hàng
      \item Phát hiện các sản phẩm thường được mua cùng nhau
    \end{itemize}
  \end{block}

  \vspace{0.5cm}

  \begin{alertblock}{Ứng dụng thực tế:}
    \begin{itemize}
      \item \textbf{Tối ưu hóa sắp đặt sản phẩm}
      \item \textbf{Cross-selling opportunities}
      \item \textbf{Tạo bundle sản phẩm}
      \item \textbf{Quản lý tồn kho}
    \end{itemize}
  \end{alertblock}
\end{frame}

\begin{frame}{TẬP DỮ LIỆU: GROCERIES}
  \begin{columns}
    \column{0.55\textwidth}
      \textbf{Thông tin tập dữ liệu:}
      \begin{itemize}
        \item Nguồn: Machine Learning with R
        \item Loại: Dữ liệu giao dịch (Transactional data)
        \item Lĩnh vực: Cửa hàng tạp hóa (Retail grocery store)
      \end{itemize}

    \column{0.45\textwidth}
      \textbf{Thống kê:}
      \begin{itemize}
        \item [Run script] transactions
        \item [Run script] unique products
        \item $\sim$4.4 items/transaction (average)
      \end{itemize}
  \end{columns}

  \vspace{0.5cm}

  \begin{exampleblock}{Cấu trúc dữ liệu:}
    Transaction 1: \{milk, bread, butter\}\\
    Transaction 2: \{beer, chips\}\\
    ...
  \end{exampleblock}
\end{frame}

\begin{frame}{THAM SỐ XUẤT HIỆN CỦA TỪNG SẢN PHẨM}
  \begin{center}
    \includegraphics[width=0.95\textwidth,height=0.7\textheight,keepaspectratio]{../../../charts/01_individual_product_occurrence.png}
  \end{center}
\end{frame}

% =============================================================================
% SECTION 2: PHƯƠNG PHÁP
% =============================================================================
\section{PHƯƠNG PHÁP: THUẬT TOÁN APRIORI}

\begin{frame}{THUẬT TOÁN APRIORI}
  \begin{block}{Nguyên tắc cơ bản:}
    \centering
    \textit{``Tất cả các tập con của một frequent itemset đều phải frequent''}
  \end{block}

  \vspace{0.3cm}

  \begin{columns}
    \column{0.5\textwidth}
      \textbf{Ưu điểm:}
      \begin{itemize}
        \item[$\checkmark$] Đơn giản, dễ hiểu
        \item[$\checkmark$] Dễ triển khai (implement)
        \item[$\checkmark$] Hiệu quả với dataset nhỏ
      \end{itemize}

    \column{0.5\textwidth}
      \textbf{Nhược điểm:}
      \begin{itemize}
        \item[$\times$] Phải quét cơ sở dữ liệu nhiều lần
        \item[$\times$] Tập ứng viên (candidate sets) lớn
        \item[$\times$] Tốn nhiều bộ nhớ
      \end{itemize}
  \end{columns}
\end{frame}

\begin{frame}{CÁC BƯỚC THUẬT TOÁN}
  \textbf{1. Khởi tạo (Initialization)}
  \begin{itemize}
    \item Thiết lập ngưỡng hỗ trợ tối thiểu (minimum support threshold)
  \end{itemize}

  \textbf{2. Sinh ứng viên (Generate Candidates)}
  \begin{itemize}
    \item Tạo k-itemsets từ (k-1)-itemsets
  \end{itemize}

  \textbf{3. Tỉa cành (Prune)}
  \begin{itemize}
    \item Loại bỏ các items có support < ngưỡng tối thiểu
    \item Áp dụng nguyên tắc Apriori (Apriori principle)
  \end{itemize}

  \textbf{4. Lặp lại (Repeat)}
  \begin{itemize}
    \item Tăng k cho đến khi không còn tìm thấy frequent items
  \end{itemize}
\end{frame}

\begin{frame}{CÁC CHỈ SỐ ĐÁNH GIÁ}
  \begin{table}
    \centering
    \small
    \begin{tabular}{|l|l|l|}
    \hline
    \textbf{Chỉ Số} & \textbf{Mô Tả} & \textbf{Công Thức} \\
    \hline
    Support & Độ hỗ trợ & P(A$\cup$B) \\
    Confidence & Độ tin cậy & P(B|A) \\
    Lift & Độ nâng & $\frac{P(A\cup B)}{P(A)P(B)}$ \\
    Leverage & Đòn bẩy & $P(A\cup B)-P(A)P(B)$ \\
    Conviction & Độ xác tín & $\frac{1-P(B)}{1-conf}$ \\
    \hline
    \end{tabular}
  \end{table}

  \vspace{0.3cm}

  \begin{itemize}
    \item \textbf{Lift > 1}: Tương quan dương $\checkmark$
    \item \textbf{Lift = 1}: Độc lập
    \item \textbf{Lift < 1}: Tương quan âm
  \end{itemize}
\end{frame}

\begin{frame}{PIPELINE XỬ LÝ DỮ LIỆU}
\centering

\begin{tikzpicture}[
    node distance=0.9cm and 1.2cm, % << giảm khoảng cách
    block/.style={
        rectangle,
        draw=blue!70!black,
        thick,
        fill=blue!20,
        rounded corners=4pt,
        text width=8.6em,
        align=center,
        inner sep=4pt,
        font=\small
    },
    line/.style={draw, -latex', thick, blue!70!black}
]

% Hàng trên
\node[block] (load) {Load Data\\\texttt{groceries.csv}};
\node[block, right=of load] (encode) {Encode\\Transaction Encoder};
\node[block, right=of encode] (clean) {Cleaning\\7-step process};

% Hàng dưới
\node[block, below=of encode] (mine) {Mining\\Apriori / FP-Growth};
\node[block, right=of mine] (rules) {Rules\\Association Rules};

% Mũi tên
\draw[line] (load) -- (encode);
\draw[line] (encode) -- (clean);
\draw[line] (clean.south) |- (mine.east);
\draw[line] (mine) -- (rules);

\end{tikzpicture}

\vspace{0.3cm}

\textbf{Tham số:}
\begin{itemize}
  \item Support tối thiểu: \textit{(run script để xác định)}
  \item Confidence tối thiểu: 0.25
\end{itemize}

\end{frame}

% =============================================================================
% SECTION 3: TRIỂN KHAI
% =============================================================================
\section{TRIỂN KHAI}

\begin{frame}{CÁC ITEMS XUẤT HIỆN THƯỜNG XUYÊN NHẤT}
  \begin{table}
    \centering
    \small
    \begin{tabular}{|l|c|c|}
    \hline
    \textbf{Sản phẩm} & \textbf{Số lượng} & \textbf{Support} \\
    \hline
    Whole milk & 2222 & 31.69\% \\
    Other vegetables & 1766 & 25.19\% \\
    Rolls/buns & 1483 & 21.15\% \\
    Soda & 1372 & 19.57\% \\
    Yogurt & 1264 & 18.03\% \\
    \hline
    \end{tabular}
  \end{table}

  \begin{block}{Nhận xét:}
    \begin{itemize}
      \item Whole milk là sản phẩm phổ biến nhất
      \item Xuất hiện trong $\sim$32\% giao dịch (transactions)
    \end{itemize}
  \end{block}
\end{frame}

\begin{frame}{TOP 20 SẢN PHẨM THƯỜNG XUYÊN NHẤT}
  \begin{center}
    \includegraphics[width=0.95\textwidth,height=0.7\textheight,keepaspectratio]{../../../charts/02_top_20_frequent_products.png}
  \end{center}
\end{frame}

\begin{frame}{PHÂN PHỐI TẦN SUẤT ITEMS}
  \begin{center}
    \includegraphics[width=0.95\textwidth,height=0.7\textheight,keepaspectratio]{../../../charts/03_item_frequency_distribution.png}
  \end{center}
\end{frame}

\begin{frame}{LUẬT KẾT HỢP}
  \textbf{Ví dụ các luật (Rules):}

  \vspace{0.3cm}

  \begin{columns}
    \column{0.5\textwidth}
      \textbf{Rule 1:} \{rolls/buns\} $\to$ \{butter\}
      \begin{itemize}
        \item Support: [value]
        \item Confidence: [value]
        \item Lift: [value]
      \end{itemize}

    \column{0.5\textwidth}
      \textbf{Rule 2:} \{milk, bread\} $\to$ \{butter\}
      \begin{itemize}
        \item Support: [value]
        \item Confidence: [value]
        \item Lift: [value]
      \end{itemize}
  \end{columns}

  \vspace{0.5cm}

  \begin{center}
    \textbf{Tổng cộng:} Phát hiện được 90 luật (rules)
  \end{center}
\end{frame}

\begin{frame}{PHÂN PHỐI TOP 10 SẢN PHẨM}
  \begin{center}
    \includegraphics[width=0.8\textwidth,height=0.65\textheight,keepaspectratio]{../../../charts/04_top_10_products_pie_chart.png}
  \end{center}
\end{frame}

\begin{frame}{SUPPORT SO VỚI CONFIDENCE}
  \begin{center}
    \includegraphics[width=0.95\textwidth,height=0.7\textheight,keepaspectratio]{../../../charts/05_support_vs_confidence_scatter.png}
  \end{center}
\end{frame}

\begin{frame}{TOP 20 LUẬT THEO LIFT}
  \begin{center}
    \includegraphics[width=0.95\textwidth,height=0.7\textheight,keepaspectratio]{../../../charts/06_top_20_rules_by_lift.png}
  \end{center}
\end{frame}

\begin{frame}{PHÂN PHỐI CÁC CHỈ SỐ CỦA LUẬT}
  \begin{center}
    \includegraphics[width=0.95\textwidth,height=0.7\textheight,keepaspectratio]{../../../charts/07_rules_metrics_distribution_boxplot.png}
  \end{center}
\end{frame}

\begin{frame}{BIỂU ĐỒ NHIỆT TƯƠNG QUAN CÁC CHỈ SỐ}
  \begin{center}
    \includegraphics[width=0.7\textwidth,height=0.7\textheight,keepaspectratio]{../../../charts/08_metrics_correlation_heatmap.png}
  \end{center}
\end{frame}

\begin{frame}{SO SÁNH HIỆU SUẤT CƠ BẢN}
  \begin{table}
    \centering
    \small
    \begin{tabular}{|l|c|c|c|}
    \hline
    \textbf{Thuật toán} & \textbf{Thời gian} & \textbf{Bộ nhớ} & \textbf{Khả năng mở rộng} \\
    \hline
    Apriori & Cơ sở & Cao & Thấp \\
    FP-Growth & 2-3x $\uparrow$ & Trung bình & Tốt \\
    FP-Max & 3-5x $\uparrow$ & Thấp & Xuất sắc \\
    \hline
    \end{tabular}
  \end{table}

  \begin{alertblock}{Kết luận:}
    \begin{itemize}
      \item FP-Growth phù hợp cho môi trường production
      \item FP-Max tốt nhất cho các dataset lớn
    \end{itemize}
  \end{alertblock}
\end{frame}

\begin{frame}{SO SÁNH THỜI GIAN THỰC THI THUẬT TOÁN}
  \begin{center}
    \includegraphics[width=0.95\textwidth,height=0.7\textheight,keepaspectratio]{../../../charts/09_algorithm_execution_time.png}
  \end{center}
\end{frame}

\begin{frame}{SỐ LƯỢNG FREQUENT ITEMSETS}
  \begin{center}
    \includegraphics[width=0.95\textwidth,height=0.7\textheight,keepaspectratio]{../../../charts/10_frequent_itemsets_count.png}
  \end{center}
\end{frame}

\begin{frame}{BIỂU ĐỒ PHÂN TÁN HIỆU SUẤT THỜI GIAN}
  \begin{center}
    \includegraphics[width=0.95\textwidth,height=0.7\textheight,keepaspectratio]{../../../charts/11_time_efficiency_scatter.png}
  \end{center}
\end{frame}

% =============================================================================
% SECTION 4: HẠN CHẾ
% =============================================================================
\section{HẠN CHẾ CỦA APRIORI}

\begin{frame}{HẠN CHẾ CỦA APRIORI}
  \textbf{1. Quét cơ sở dữ liệu nhiều lần}
  \begin{itemize}
    \item Cần K lần quét cho k-itemsets
    \item Tốn nhiều thao tác I/O
  \end{itemize}

  \textbf{2. Tập ứng viên lớn (Large Candidate Sets)}
  \begin{itemize}
    \item Tăng trưởng theo cấp số nhân
    \item Có thể có 2$^k$ - 1 itemsets
  \end{itemize}

  \textbf{3. Sử dụng bộ nhớ (Memory Usage)}
  \begin{itemize}
    \item Phải lưu tất cả các ứng viên (candidates)
    \item Khó mở rộng (scale) với dataset lớn
  \end{itemize}

  \textbf{4. Chi phí tính toán (Computational Cost)}
  \begin{itemize}
    \item Sinh ứng viên (candidate generation) tốn kém
    \item Đếm support rất tốn kém
  \end{itemize}
\end{frame}

% =============================================================================
% SECTION 5: CÁCH CẢI THIỆN
% =============================================================================
\section{CÁCH CẢI THIỆN}

\begin{frame}{CẢI TIẾN 1: SAMPLING}
  \begin{block}{Ý tưởng:}
    \begin{itemize}
      \item Khai phá trên mẫu dữ liệu (30\%)
      \item Xác minh trên toàn bộ dataset
      \item Nhanh cho phân tích khám phá (exploratory analysis)
    \end{itemize}
  \end{block}

  \textbf{Triển khai:}
  \texttt{sampling\_based\_fim(df, min\_support, sample\_ratio=0.3)}
\end{frame}

\begin{frame}{CẢI TIẾN 2: DHP (HASH-BASED)}
  \begin{block}{Ý tưởng:}
    \begin{itemize}
      \item Sử dụng hash để tỉa bớt candidates
      \item Chỉ cần 2-3 lần quét cơ sở dữ liệu
      \item Giảm candidates nhanh hơn
    \end{itemize}
  \end{block}

  \textbf{Triển khai:}
  \texttt{dhp\_algorithm(transactions, min\_support, hash\_table\_size)}
\end{frame}

\begin{frame}{CẢI TIẾN 3: TRANSACTION REDUCTION}
  \begin{block}{Ý tưởng:}
    \begin{itemize}
      \item Loại bỏ các giao dịch không chứa frequent items
      \item Giảm kích thước cơ sở dữ liệu qua các vòng lặp
      \item Tiết kiệm bộ nhớ
    \end{itemize}
  \end{block}

  \textbf{Triển khai:}
  \texttt{transaction\_reduction\_apriori(df, min\_support)}
\end{frame}

\begin{frame}{CẢI TIẾN 4: ECLAT (VERTICAL)}
  \begin{block}{Ý tưởng:}
    \begin{itemize}
      \item Sử dụng định dạng tid-lists dọc (vertical)
      \item Phép giao (intersection) nhanh
      \item Xuất sắc cho dataset thưa (sparse)
    \end{itemize}
  \end{block}

  \textbf{Triển khai:}
  \texttt{eclat\_algorithm(df, min\_support)}
\end{frame}

\begin{frame}{CẢI TIẾN 5: DIC (DYNAMIC COUNTING)}
  \begin{block}{Ý tưởng:}
    \begin{itemize}
      \item Đếm xen kẽ (interleaved counting)
      \item Ít lần quét cơ sở dữ liệu hơn
    \end{itemize}
  \end{block}

  \textbf{Triển khai:}
  \texttt{dic\_algorithm(df, min\_support)}
\end{frame}

\begin{frame}{CẢI TIẾN 6: PARTITIONING}
  \begin{block}{Ý tưởng:}
    \begin{itemize}
      \item Chia cơ sở dữ liệu thành các phân vùng (partitions)
      \item Khai phá cục bộ, xác minh toàn cục
    \end{itemize}
  \end{block}

  \textbf{Triển khai:}
  \texttt{partitioning\_apriori(df, min\_support, n\_partitions=5)}
\end{frame}

% =============================================================================
% SECTION 6: KHUYẾN NGHỊ
% =============================================================================
\section{KHUYẾN NGHỊ SỬ DỤNG}

\begin{frame}{KHUYẾN NGHỊ SỬ DỤNG}
  \begin{columns}
    \column{0.33\textwidth}
      \begin{block}{Nhỏ (< 10K)}
        Apriori hoặc ECLAT\\
        \small{Đơn giản, dễ hiểu}
      \end{block}

    \column{0.33\textwidth}
      \begin{block}{Trung bình (10K-1M)}
        FP-Growth hoặc DIC\\
        \small{Hiệu suất tốt}
      \end{block}

    \column{0.33\textwidth}
      \begin{block}{Lớn (> 1M)}
        FP-Max, Partitioning\\
        \small{Dễ mở rộng}
      \end{block}
  \end{columns}

  \vspace{0.5cm}

  \begin{itemize}
    \item \textbf{Thời gian thực/Streaming}: Các phương pháp dựa trên sampling
    \item \textbf{Dataset thưa (Sparse)}: ECLAT (định dạng vertical rất hiệu quả)
    \item \textbf{Bộ nhớ hạn chế}: Transaction Reduction hoặc Partitioning
  \end{itemize}
\end{frame}

% =============================================================================
% SECTION 7: KẾT LUẬN
% =============================================================================
\section{KẾT LUẬN}

\begin{frame}{KẾT LUẬN}
  \begin{block}{Điểm chính:}
    \begin{itemize}
      \item Apriori: Nền tảng cho khai phá frequent itemsets
      \item Đã triển khai 6 thuật toán cải tiến
      \item Giảm 30-50\% thời gian thực thi với các kỹ thuật tối ưu
      \item FP-Growth/FP-Max phù hợp cho môi trường production
    \end{itemize}
  \end{block}

  \vspace{0.3cm}

  \textbf{Các thuật toán đã triển khai:}
  \begin{itemize}
    \item Sampling, DHP, Transaction Reduction
    \item ECLAT, DIC, Partitioning
  \end{itemize}

  \textbf{Ứng dụng thực tế:}
  \begin{itemize}
    \item Bán lẻ (Retail) \& Thương mại điện tử, Chăm sóc sức khỏe, Khai phá sử dụng web, Tin sinh học
  \end{itemize}
\end{frame}

\begin{frame}
  \begin{center}
    \Huge \textbf{CẢM ƠN ĐÃ LẮNG NGHE}


  \end{center}
\end{frame}

\end{document}
