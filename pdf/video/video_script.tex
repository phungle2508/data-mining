% !TEX program = pdflatex
% Video Script - Cải Thiện Apriori
% Data Mining Project 2024-2025

\documentclass[a4paper,12pt]{article}

% Packages for Vietnamese language support
\usepackage[utf8]{inputenc}
\usepackage[T5]{fontenc}
\usepackage[vietnamese]{babel}

% Fonts
\usepackage{mathptmx}
\usepackage{helvet}
\usepackage{amssymb}

% Page layout
\usepackage[a4paper,
            top=1in,
            bottom=1in,
            left=1in,
            right=1in]{geometry}

% Line spacing
\usepackage{setspace}
\onehalfspacing

% Other packages
\usepackage{titlesec}
\usepackage{fancyhdr}
\usepackage{graphicx}
\usepackage{xcolor}
\usepackage{hyperref}
\usepackage{enumitem}
\usepackage{tcolorbox}

% Hyperlink setup
\hypersetup{
    colorlinks=true,
    linkcolor=blue,
    urlcolor=cyan,
}

% Page style
\pagestyle{fancy}
\fancyhf{}
\fancyhead[L]{\small Video Script - Data Mining}
\fancyhead[R]{\small Cải Thiện Apriori}
\fancyfoot[C]{\thepage}

% Title formatting
\titleformat{\section}
  {\normalfont\Large\bfseries\color{blue}}{\thesection}{1em}{}
\titleformat{\subsection}
  {\normalfont\large\bfseries\color{darkgray}}{\thesubsection}{1em}{}

% Document info
\title{\textbf{VIDEO PRESENTATION SCRIPT}\\
\large Data Mining: Cải Thiện Apriori\\
\small Duration: 5-10 minutes}

\author{Nhóm [Tên nhóm]}
\date{Năm học 2024 - 2025}

\begin{document}

% Title page
\maketitle
\thispagestyle{empty}

\begin{center}
  \large \textbf{Video Recording Requirements}
\end{center}

\begin{itemize}
  \item \textbf{Duration}: 5-10 minutes
  \item \textbf{Show presenter}: Yes (hiện hình người báo cáo)
  \item \textbf{Audio quality}: Clear and audible
  \item \textbf{Screen sharing}: Clear and visible
\end{itemize}

\newpage

% =============================================================================
% [00:00 - 00:45] INTRODUCTION
% =============================================================================
\section{[00:00 - 00:45] INTRODUCTION}

\begin{tcolorbox}[colback=blue!5!white, colframe=blue!75!black, title=Visual]
Title slide with ``CẢI THIỆN APRIORI - Frequent Itemset Mining và Association Rules''
\end{tcolorbox}

\subsection{Speaker Script:}

\begin{quote}
``Xin chào thầy và các bạn. Hôm nay nhóm xin trình bày về đề tài \textbf{'Cải thiện Apriori'} trong môn Data Mining. Đề tài tập trung vào việc phân tích Market Basket và tìm các cải tiến cho thuật toán Apriori cổ điển.''
\end{quote}

\textbf{Key Points:}
\begin{itemize}
  \item Greeting and introduction
  \item Project title clearly stated
  \item Focus on Market Basket Analysis
  \item Mention Apriori algorithm improvements
\end{itemize}

\textbf{Body Language:}
\begin{itemize}
  \item Smile and maintain eye contact with camera
  \item Speak clearly and at moderate pace
  \item Use hand gestures to emphasize key points
\end{itemize}

\newpage

% =============================================================================
% [00:45 - 01:30] PROBLEM STATEMENT
% =============================================================================
\section{[00:45 - 01:30] PROBLEM STATEMENT}

\begin{tcolorbox}[colback=green!5!white, colframe=green!75!black, title=Visual]
Slide showing Market Basket Analysis applications
\end{tcolorbox}

\subsection{Speaker Script:}

\begin{quote}
``Vấn đề chúng ta giải quyết là làm sao để hiểu rõ hành vi mua sắm của khách hàng thông qua việc phân tích dữ liệu giao dịch. Từ đó, có thể phát hiện các sản phẩm thường được mua cùng nhau và đưa ra các gợi ý phù hợp.''
\end{quote}

\begin{quote}
``Các ứng dụng thực tế bao gồm: tối ưu hóa sắp đặt sản phẩm, cross-selling opportunities, tạo bundle sản phẩm, và quản lý tồn kho hiệu quả hơn.''
\end{quote}

\textbf{Key Points:}
\begin{itemize}
  \item Problem statement: Understanding customer shopping behavior
  \item Goal: Discover products purchased together
  \item Real-world applications mentioned
\end{itemize}

\textbf{Screen Share:}
\begin{itemize}
  \item Show the Market Basket Analysis slide
  \item Point to each application as you mention it
\end{itemize}

\newpage

% =============================================================================
% [01:30 - 02:30] DATASET
% =============================================================================
\section{[01:30 - 02:30] DATASET}

\begin{tcolorbox}[colback=yellow!5!white, colframe=yellow!75!black, title=Visual]
Groceries dataset statistics
\end{tcolorbox}

\subsection{Speaker Script:}

\begin{quote}
``Dataset chúng tôi sử dụng là Groceries dataset, một bộ dữ liệu từ Machine Learning with R. Đây là dữ liệu giao dịch từ một cửa hàng tạp hóa, nơi mỗi giao dịch đại diện cho một giỏ hàng của khách hàng.''
\end{quote}

\begin{quote}
``Dataset chứa [số lượng sau khi chạy script.py] giao dịch và [số lượng sau khi chạy script.py] sản phẩm khác nhau. Trung bình mỗi khách hàng mua khoảng 4.4 sản phẩm trong một lần giao dịch.''
\end{quote}

\textbf{Key Points:}
\begin{itemize}
  \item Dataset source: Machine Learning with R
  \item Type: Transactional data from grocery store
  \item Each transaction = customer basket
  \item Statistics: [Run script for actual values]
\end{itemize}

\textbf{Screen Share:}
\begin{itemize}
  \item Show the Groceries dataset statistics slide
  \item Highlight key numbers
\end{itemize}

\newpage

% =============================================================================
% [02:30 - 04:00] APRIORI ALGORITHM
% =============================================================================
\section{[02:30 - 04:00] APRIORI ALGORITHM}

\begin{tcolorbox}[colback=red!5!white, colframe=red!75!black, title=Visual]
Apriori algorithm flow and steps
\end{tcolorbox}

\subsection{Speaker Script:}

\begin{quote}
``Apriori là thuật toán cổ điển được đề xuất bởi Agrawal và Srikant vào năm 1994. Nó dựa trên nguyên tắc: tất cả các tập con của một frequent itemset cũng phải là frequent.''
\end{quote}

\begin{quote}
``Thuật toán có 4 bước chính:

\textbf{1. Initialization}: Thiết lập minimum support threshold

\textbf{2. Generate Candidates}: Tạo k-itemsets từ (k-1)-itemsets

\textbf{3. Prune}: Loại bỏ items có support thấp hơn ngưỡng

\textbf{4. Repeat}: Tăng k và lặp lại cho đến khi không tìm thấy frequent items nào nữa''
\end{quote}

\begin{quote}
``Chúng tôi cũng sử dụng các chỉ số đánh giá như Support, Confidence, Lift, Leverage, Conviction, và Zhang's Metric để đánh giá chất lượng các association rules.''
\end{quote}

\textbf{Key Points:}
\begin{itemize}
  \item Algorithm: Apriori (1994, Agrawal \& Srikant)
  \item Core principle: Subsets of frequent itemsets are also frequent
  \item 4 main steps explained
  \item Evaluation metrics mentioned
\end{itemize}

\textbf{Screen Share:}
\begin{itemize}
  \item Show the algorithm flow diagram
  \item Highlight each step as you explain it
  \item Show the evaluation metrics table
\end{itemize}

\newpage

% =============================================================================
% [04:00 - 05:30] IMPLEMENTATION \& RESULTS
% =============================================================================
\section{[04:00 - 05:30] IMPLEMENTATION \& RESULTS}

\begin{tcolorbox}[colback=cyan!5!white, colframe=cyan!75!black, title=Visual]
Pipeline processing and results
\end{tcolorbox}

\subsection{Speaker Script:}

\begin{quote}
``Chúng tôi implement thuật toán sử dụng Python và thư viện mlxtend. Pipeline xử lý bao gồm: Load Data, Encode, Cleaning, Mining, và Rules Generation.''
\end{quote}

\begin{quote}
``Kết quả cho thấy Whole milk là sản phẩm phổ biến nhất với khoảng 25.5\% support. Các sản phẩm phổ biến tiếp theo bao gồm Other vegetables, Rolls/buns, Soda, và Yogurt.''
\end{quote}

\begin{quote}
``Chúng tôi tìm được [số lượng sau khi chạy script.py] association rules với minimum confidence là 0.25. Các rule này cho thấy các mối quan hệ thú vị giữa các sản phẩm.''
\end{quote}

\textbf{Key Points:}
\begin{itemize}
  \item Implementation: Python + mlxtend
  \item Pipeline: 5 steps
  \item Top product: Whole milk (~25.5\% support)
  \item Rules discovered: [Run script for actual value]
  \item Min confidence: 0.25
\end{itemize}

\textbf{Screen Share:}
\begin{itemize}
  \item Show the pipeline diagram
  \item Display top frequent items table
  \item Show example association rules
\end{itemize}

\newpage

% =============================================================================
% [05:30 - 07:30] IMPROVEMENT STRATEGIES
% =============================================================================
\section{[05:30 - 07:30] IMPROVEMENT STRATEGIES}

\begin{tcolorbox}[colback=magenta!5!white, colframe=magenta!75!black, title=Visual]
6 improvement strategies
\end{tcolorbox}

\subsection{Speaker Script:}

\begin{quote}
``Apriori có 4 hạn chế chính: multiple database scans, large candidate sets, high memory usage, và expensive computational cost. Để giải quyết các vấn đề này, chúng tôi đã triển khai 6 chiến lược cải tiến:''
\end{quote}

\begin{enumerate}
  \item \textbf{Sampling}: Mine trên sample 30\% dữ liệu để giảm chi phí tính toán
  \item \textbf{DHP (Hash-based)}: Sử dụng hash table để pruning candidates hiệu quả hơn
  \item \textbf{Transaction Reduction}: Loại bỏ transactions không chứa frequent items
  \item \textbf{ECLAT (Vertical)}: Sử dụng vertical tid-list format để nhanh hơn
  \item \textbf{DIC (Dynamic Counting)}: Interleaved counting để giảm database scans
  \item \textbf{Partitioning}: Chia database thành các partitions để xử lý độc lập
\end{enumerate}

\textbf{Key Points:}
\begin{itemize}
  \item 4 main limitations of Apriori
  \item 6 improvement strategies implemented
  \item Each strategy briefly explained
\end{itemize}

\textbf{Screen Share:}
\begin{itemize}
  \item Show the limitations slide
  \item Display each improvement strategy
  \item Use diagrams to illustrate concepts
\end{itemize}

\newpage

% =============================================================================
% [07:30 - 08:30] COMPARISON
% =============================================================================
\section{[07:30 - 08:30] COMPARISON}

\begin{tcolorbox}[colback=orange!5!white, colframe=orange!75!black, title=Visual]
Algorithm comparison table
\end{tcolorbox}

\subsection{Speaker Script:}

\begin{quote}
``So sánh 9 thuật toán cho thấy FP-Growth và FP-Max có hiệu suất tốt nhất. Với Groceries dataset:''
\end{quote}

\begin{quote}
``- FP-Growth nhanh hơn 2-3 lần so với Apriori truyền thống

- FP-Max nhanh hơn 3-5 lần và tối ưu bộ nhớ tốt hơn

- Các cải tiến đề xuất có thể giảm 30-50\% execution time''
\end{quote}

\begin{quote}
``Kết quả này cho rằng Apriori vẫn là nền tảng vững chắc, nhưng các thuật toán cải tiến hoặc các phương pháp tiếp cận mới như FP-Growth và FP-Max là lựa chọn tốt hơn cho production environments.''
\end{quote}

\textbf{Key Points:}
\begin{itemize}
  \item FP-Growth: 2-3x faster
  \item FP-Max: 3-5x faster, better memory optimization
  \item Proposed improvements: 30-50\% time reduction
  \item Recommendation: FP-Growth/FP-Max for production
\end{itemize}

\textbf{Screen Share:}
\begin{itemize}
  \item Show the comparison table
  \item Highlight performance improvements
  \item Display performance metrics
\end{itemize}

\newpage

% =============================================================================
% [08:30 - 09:30] CONCLUSION
% =============================================================================
\section{[08:30 - 09:30] CONCLUSION}

\begin{tcolorbox}[colback=purple!5!white, colframe=purple!75!black, title=Visual]
Summary and applications
\end{tcolorbox}

\subsection{Speaker Script:}

\begin{quote}
``Tóm lại:''
\end{quote}

\begin{quote}
``- Apriori là nền tảng vững chắc cho frequent itemset mining

- 6 thuật toán cải tiến đã được triển khai và đánh giá

- Các kỹ thuật tối ưu có thể giảm 30-50\% execution time

- FP-Growth và FP-Max là lựa chọn tốt nhất cho production environments''
\end{quote}

\begin{quote}
``Ứng dụng thực tế của các thuật toán này rất rộng:''
\end{quote}

\begin{itemize}
  \item \textbf{Retail và E-commerce}: Market Basket Analysis
  \item \textbf{Healthcare}: Pattern detection trong bệnh sử
  \item \textbf{Web usage mining}: Phân tích hành vi người dùng
  \item \textbf{Bioinformatics}: Phân tích gene sequences và protein interactions
\end{itemize}

\textbf{Key Points:}
\begin{itemize}
  \item Summary of achievements
  \item Performance improvements
  \item Real-world applications
\end{itemize}

\textbf{Screen Share:}
\begin{itemize}
  \item Show summary slide
  \item Display application areas
\end{itemize}

\newpage

% =============================================================================
% [09:30 - 10:00] Q\&A
% =============================================================================
\section{[09:30 - 10:00] Q\&A}

\begin{tcolorbox}[colback=gray!5!white, colframe=gray!75!black, title=Visual]
``Thank you! Questions?'' slide
\end{tcolorbox}

\subsection{Speaker Script:}

\begin{quote}
``Cảm ơn thầy và các bạn đã lắng nghe. Nhóm xin nhận câu hỏi.''
\end{quote}

\textbf{Key Points:}
\begin{itemize}
  \item Thank the audience
  \item Invite questions
  \item Be prepared for:
  \begin{itemize}
    \item Algorithm complexity questions
    \item Implementation details
    \item Dataset characteristics
    \item Performance metrics
  \end{itemize}
\end{itemize}

\textbf{Body Language:}
\begin{itemize}
  \item Smile and appear approachable
  \item Maintain eye contact
  \item Listen carefully to questions
  \item Answer concisely and clearly
\end{itemize}

\newpage

% =============================================================================
% RECORDING TIPS
% =============================================================================
\section{RECORDING TIPS}

\subsection{Before Recording:}
\begin{enumerate}
  \item Practice the script 2-3 times to get comfortable
  \item Check microphone and camera setup
  \item Ensure good lighting (natural light is best)
  \item Clean and professional background
\end{enumerate}

\subsection{During Recording:}
\begin{enumerate}
  \item \textbf{Eye contact}: Look at the camera, not the screen
  \item \textbf{Speaking pace}: Moderate and clear
  \item \textbf{Pronunciation}: Clear and articulate
  \item \textbf{Gestures}: Natural hand movements to emphasize points
  \item \textbf{Screen sharing}: Make sure the code/slides are clearly visible
\end{enumerate}

\subsection{Technical Setup:}
\begin{itemize}
  \item Use a quiet room
  \item Good internet connection (if streaming)
  \item Microphone close but not too close
  \item Camera at eye level
  \item Screen resolution: 1920x1080 or higher
\end{itemize}

\subsection{Post-Recording:}
\begin{enumerate}
  \item Check audio quality
  \item Verify screen clarity
  \item Edit out any major mistakes
  \item Add captions if possible
  \item Keep final video under 10 minutes
\end{enumerate}

\newpage

\section{CHECKLIST}

\begin{itemize}
  \item[$\square$] Practice script 2-3 times
  \item[$\square$] Set up proper lighting
  \item[$\square$] Test microphone quality
  \item[$\square$] Clean background
  \item[$\square$] Check internet connection
  \item[$\square$] Prepare slides for screen sharing
  \item[$\square$] Have water nearby
  \item[$\square$] Test recording software
  \item[$\square$] Record test video
  \item[$\square$] Check video quality before final recording
\end{itemize}

\end{document}
